\documentclass[spanish, 10pt,a4paper]{article}
\usepackage[spanish]{babel}
\usepackage[utf8]{inputenc}
\usepackage{textcomp}
\usepackage{hyperref}
\usepackage[pdftex]{graphicx}
\usepackage{epsfig}
\usepackage{amsmath}
\usepackage{hyperref}
\usepackage{amssymb}
\usepackage{color}
\usepackage{graphics}
\usepackage{amsthm}
\usepackage{subcaption}
\usepackage{caratula}
\usepackage{fancyhdr,lastpage}
\usepackage[paper=a4paper, left=1.4cm, right=1.4cm, bottom=1.4cm, top=1.4cm]{geometry}
\usepackage[table]{xcolor} % color en las matrices
\usepackage[font=small,labelfont=bf]{caption} % caption de las figuras en letra mas chica que el texto
\usepackage[ruled,vlined,linesnumbered]{algorithm2e}
\usepackage{listings}
\usepackage{float}
\usepackage{amsfonts}
\usepackage{upgreek}

\color{black}

%%%PAGE LAYOUT%%%
\topmargin = -1.2cm
\voffset = 0cm
\hoffset = 0em
\textwidth = 48em
\textheight = 164 ex
\oddsidemargin = 0.5 em
\parindent = 2 em
\parskip = 3 pt
\footskip = 7ex
\headheight = 20pt
\pagestyle{fancy}
\lhead{Paradigmas y lenguajes de programaci\'on - 2014 C2 - Trabajo Pr\'actico 1} % cambia la parte izquierda del encabezado
\renewcommand{\sectionmark}[1]{\markboth{#1}{}} % cambia la parte derecha del encabezado
\rfoot{\thepage}
\cfoot{}

%spaced sections



%El siguiente paquete permite escribir la caratula facilmente
\hypersetup{
  pdftitle={ 2c2014.PLP.TP1},
  colorlinks,
  citecolor=black,
  filecolor=black,
  linkcolor=black,
  urlcolor=black 
}

\materia{Paradigmas y lenguajes de programaci\'on}

\titulo{Trabajo Pr\'actico 1}

\subtitulo{Programaci\'on funcional.}

\grupo{Grupo: Java decime que se siente}

\integrante{Leandro Matayoshi}{79/11}{leandro.matayoshi@gmail.com}
\integrante{Ignacio Niesz}{722/10}{ignacio.niesz@gmail.com}
\integrante{Raul Benitti}{592/08}{raulbenitti@gmail.com}
 
\begin{document}
{ \oddsidemargin = 2em
	\headheight = -20pt
	\maketitle
}
	
	\section*{Oogle Amps}

\lstset{
  frame=none,
  xleftmargin=2pt,
  stepnumber=1,
  numbers=left,
  numbersep=5pt,
  numberstyle=\ttfamily\tiny\color[gray]{0.3},
  belowcaptionskip=\bigskipamount,
  captionpos=b,
  escapeinside={*'}{'*},
  language=haskell,
  tabsize=2,
  emphstyle={\bf},
  commentstyle=\it,
  stringstyle=\mdseries\rmfamily,
  showspaces=false,
  keywordstyle=\bfseries\rmfamily,
  columns=flexible,
  basicstyle=\small\sffamily,
  showstringspaces=false,
  morecomment=[l]\%,
}

\subsubsection*{Ejercicio 1}
\begin{lstlisting}
-- Define una funcion que recibe una clave y un diccionario. Esta funcion utiliza filter
-- para filtrar la clave buscada de las claves del diccionario. La funcion se asegura
-- que la lista de claves filtradas no sea vacio.
belongs :: Eq k => k -> Dict k v -> Bool
belongs = \k dict -> not ( null (filter (\e -> fst e == k) dict) )

-- Invierte el orden en que belongs recibe los parametros para que tenga la funcionalidad
-- pedida.
(?) :: Eq k => Dict k v -> k -> Bool
(?) = flip belongs 


--Main> [("calle",[3]),("city",[2,1])] ? "city" 
--True
\end{lstlisting}
\vspace{1cm}

\subsubsection*{Ejercicio 2}
\begin{lstlisting}
-- Define una funcion que recibe una clave y un diccionario. Esta funcion utiliza filter
-- para filtrar de las tuplas que componen el diccionario, la que concuerdo con la clave
-- buscada.
get :: Eq k => k -> Dict k v -> v
get = \k dict -> snd ((filter (\(x,y) -> x==k) dict) !! 0)


-- Invierte el orden en que get recibe los parametros para que tenga la funcionalidad
-- pedida.
(!) :: Eq k => Dict k v -> k -> v
(!) = flip get


--Main> [("calle",[3]),("city",[2,1])] ! "city" 
--[2,1]
\end{lstlisting}
\vspace{1cm}

\subsubsection*{Ejercicio 3}
\begin{lstlisting}
-- Define una funcion que recibe una funcion para combinar los valores del diccionario
-- que poseen una misma clave, una clave del diccionario y un diccionario. La funcion se 
-- asegura de modificar el diccionario en caso de que la clave exista o de insertar 
-- un nuevo elemento en caso contrario.
insertWith :: Eq k => (v -> v -> v) -> k -> v -> Dict k v -> Dict k v
insertWith = \f k v dict -> if dict ? k then 
								updateDict f k v dict 
							else 
								insertInDict k v dict
										
-- Utiliza foldr para componer el valor insertado que corresponde a una clave ya existente.
-- El foldr se define recursivamente sobre los elementos del diccionario (una lista de tuplas,
-- (clave, valor)) asegurandose de llamar a la funcion para componer el resultado sobre los
-- elementos de la lista que tengan la misma clave que la pasada por parametro. El resto
-- de los elementos no se alteran.
updateDict :: Eq k => (v -> v -> v) -> k -> v -> Dict k v -> Dict k v
updateDict f k v dict = foldr (\e rec -> if fst e == k then (k, (f (snd e) v )) : rec else e:rec ) [] dict 

-- Simplemente se agrega el nuevo elemento al diccionario.
insertInDict :: Eq k => k -> v -> Dict k v -> Dict k v
insertInDict k v dict = dict ++ [(k,v)] 


--Main> insertWith (++) 2 ['p'] (insertWith (++) 1 ['a','b'] (insertWith (++) 1 ['l'] []))
--[(1,"lab"),(2,"p")]
\end{lstlisting}
\vspace{1cm}

\subsubsection*{Ejercicio 4}
\begin{lstlisting}
-- Define una funcion que recibe una lista de tuplas (clave, valor) y devuelve un diccionario.
-- La funcion se define mediante foldl que recibe la funcion insertWith para generar recursivamente
-- un diccionario en base a los elementos de la lista de tuplas. Notemos que en caso de coliciones
-- en las claves utilizara la concatenacion como funcion para componer los valores (agrupandolos
-- por clave).
groupByKey :: Eq k => [(k,v)] -> Dict k [v]
groupByKey = \l -> foldl (\rec (k,v)-> insertWith (++) k [v] rec) [] l


--Main> groupByKey [(1,"lab"), (1,"lab"),(2,"p")]
--[(1,["lab","lab"]),(2,["p"])]
\end{lstlisting}
\vspace{1cm}


\subsubsection*{Ejercicio 5}
\begin{lstlisting}
-- Define una funcion que recibe una funcion de "valores, en valores en valores" y dos diccionarios.
-- La funcion se define mediante foldl que recibe la funcion insertWith que a su vez recibe la
-- funcion utilizada para componer los valores de los diccionarios. El foldl realiza recursion sobre
-- la concatenacion de ambos diccionarios generando como resultado un nuevo diccionario construido
-- a partir de los insertWith con su respectiva funcion para componer valores.
unionWith :: Eq k => (v -> v -> v) -> Dict k v -> Dict k v -> Dict k v
unionWith = \f d1 d2 -> foldl (\rec (k, v) -> insertWith f k v rec) [] (d1 ++ d2) 


--Main> unionWith (++) [("calle",[3]),("city",[2,1])] [("calle", [4]), ("altura", [1,3,2])]
--[("calle",[3,4]),("city",[2,1]),("altura",[1,3,2])]
\end{lstlisting}
\vspace{1cm}

\subsubsection*{Ejercicio 6}
\begin{lstlisting}
-- Se define mediante foldl. foldl recibe una funcion encargada de tomar la cola de la recursion
-- y concatenarla la cabeza a la que le agrega un elemento. foldl realiza la recursion sobre la lista
-- de procesos y tiene como caso base una lista de listas con tantos elementos como procesadores a utilizar.
-- De esta manera recorre cada elemento rotando las listas a las cuales debe agregarlos.
-- La idea subyacente es la siguiente: La funcion combinadora se encarga de mantener el resultado recursivo
-- ordenado de manera creciente en cantidad de elementos (en la primera posicion se ubicara siempre
-- el elemento con menor cantidad de elementos). De esta manera, luego de insertar un elemento en la
-- lista que ocupa la primera posicion, la misma debe ser enviada al final para que los nuevos elementos
-- sean insertados en las listas siguientes.
-- Ejemplo con distribucion de elementos en 3 listas:
-- Inicio: [1,2,3,4] [ [], [], [] ]
-- Paso 1: [2,3,4] [ [], [], [1] ]
-- Paso 2: [3,4] [ [], [1], [2] ]
-- Paso 3: [4] [ [1], [2], [3] ]
-- Paso 4: [ [2], [3], [1,4] ]

distributionProcess :: Int -> [a] -> [[a]]
distributionProcess n l = foldl (\rec e -> (tail rec) ++ [(head rec) ++ [e]] ) (replicate n []) l


--Main> distributionProcess 2 ["p1", "p2", "p3", "p4", "p5", "p6", "p7", "p8", "p9", "p10"]
-- [["p1","p3","p5","p7","p9"],["p2","p4","p6","p8","p10"]]

--Main> distributionProcess 3 ["p1", "p2", "p3", "p4", "p5", "p6", "p7", "p8", "p9", "p10"]
--  [["p2","p5","p8"],["p3","p6","p9"],["p1","p4","p7","p10"]]
\end{lstlisting}
\vspace{1cm}

\subsubsection*{Ejercicio 7}
\begin{lstlisting}
-- Se define mediante la aplicacion de la funcion de mapeo (Mapper) utilizando map sobre la lista
-- de valores y concatenando los resultados (la lista de listas de tuplas clave valor). A su
-- vez este resultado se agrupa por clave utilizando groupByKey.
mapperProcess :: Eq k => Mapper a k v -> [a] -> [(k,[v])]
mapperProcess f l = groupByKey ( concat (map f l) )


--Main> mapperProcess (\x -> [(show x, x)]) [1, 1, 1, 2, 3, 3, 4, 5, 5, 5, 5, 5, 5, 5, 5]
-- [("1",[1,1,1]),("2",[2]),("3",[3,3]),("4",[4]),("5",[5,5,5,5,5,5,5,5])]
\end{lstlisting}
\vspace{1cm}

\subsubsection*{Ejercicio 8}
\begin{lstlisting}
-- Se define como la aplicacion de map sobre la concatenacion de la lista de listas, agrupadas 
-- y ordenadas por clave. Al agrupar por clave, se generan tuplas (k, [[v]]). Finalmente el map
-- se encarga de enviar cada tupla (k, [[v]]) a la correspondiente (k, [v]).
combinerProcess :: (Eq k, Ord k) => [[(k, [v])]] -> [(k,[v])]
combinerProcess ls = map (\(k, l) -> (k, concat l)) (sortByKey (groupByKey ( concat ls ))) 


--Main> combinerProcess [[("1",[1,1,1]), ("2",[2,2])], [("3",[3,3,3,3,3,3])], [("1",[1,1,1])]]
-- [("1",[1,1,1,1,1,1]),("2",[2,2]),("3",[3,3,3,3,3,3])]


-- Se define como el sortBy de la lista de tuplas (k, [v]) utilizando como funcion de comparacion
-- una funcion que compara entre claves de tuplas.
sortByKey::(Eq k, Ord k) => [(k, [v])] -> [(k,[v])]
sortByKey = sortBy ( (\e1 e2 -> compare (fst e1) (fst e2) ) )
\end{lstlisting}
\vspace{1cm}

\subsubsection*{Ejercicio 9}
\begin{lstlisting}
-- Se define como la concatenacion del resultado de la funcion map que aplica la funcion "Reducer"
-- a todos los elementos de la lista de tuplas (k, [v]).
reducerProcess :: Reducer k v b -> [(k, [v])] -> [b]
reducerProcess f l = concat ( map f l)


-- -Main> reducerProcess (\(k, v) -> [(k, sum v) ]) [("1",[1,1,1,1,1,1]),("2",[2,2]),("3",[3,3,3,3,3,3])]
-- [("1",6),("2",4),("3",18)]
\end{lstlisting}
\vspace{1cm}

\subsubsection*{Ejercicio 10}
\begin{lstlisting}
-- Primero se mapea la funcion resultante de aplicar mapperProcess a la funcion mapper a la lista de listas obtenia
-- al utilizar distributionProcess en 100 maquinas. Luego se combinan los resultados con combinerProcess y finalmente
-- se aplica el reducer utilizando el reducerProcess.
mapReduce :: (Eq k, Ord k) => Mapper a k v -> Reducer k v b -> [a] -> [b]
mapReduce mapper reducer l =  reducerProcess reducer (
				combinerProcess (map (mapperProcess mapper) (distributionProcess 100 l) ) 
					)


-- Main> mapReduce   (\x -> [(show x, x)]) (\(k, v) -> [(k, length v) ]) [1, 1, 1, 2, 3, 3, 4, 5, 5, 5, 5, 5, 5, 5, 5]
-- [("1",3),("2",1),("3",2),("4",1),("5",8)]
\end{lstlisting}
\vspace{1cm}


\subsubsection*{Ejercicio 11}
\begin{lstlisting}
-- Se define aplicando mapReduce al mapper, reducer y la lista de monumentos. El mapper mapea cada elemento como
-- una tupla con (nombre, cantidad) que luego se reducen agrupando la cantidad total por clave.
visitasPorMonumento :: [String] -> Dict String Int
visitasPorMonumento = \lin -> mapReduce mapper reducer lin
 where mapper = \e -> [(e, 1)]
       reducer = \(k, l) -> [(k, (length l))]


-- Main> visitasPorMonumento ["m1", "m1", "m1", "m1", "m1", "m2", "m2", "m2", "m3", "m4"]
-- [("m1",5),("m2",3),("m3",1),("m4",1)]
\end{lstlisting}
\vspace{1cm}

\subsubsection*{Ejercicio 12}
\begin{lstlisting}
-- Aprovecha que las claves se devuelven ordenadas crecientemente y el mapper utiliza el valor negativo
-- de las visitas como clave para ordenarlas por mas visitados. 
-- Luego las claves se descartan y se devuelven los valores.
-- Idea-ejemplo:
--	["m1", "m2", "m2"] 
--		visitasPorMonumentos --> [("m1", 1),("m2", 2)]
--			mapper->[(-1, "m1"), (-2, "m2")] 
--				orderByKey --> [(-2, "m2"), (-1, "m1")] 
--					reducecer -> ["m2", "m1"]
monumentosTop:: [String] -> [String]
monumentosTop = \ls -> mapReduce mapper reducer ( visitasPorMonumento ls ) 
	where mapper = \(k, v) -> [(-v, k)]
	      reducer = \(k, l) -> l

-- Main>  monumentosTop ["m1", "m1", "m1", "m1", "m1", "m2", "m2", "m2", "m3", "m4"]
-- ["m1","m2","m3","m4"]
\end{lstlisting}
\vspace{1cm}

\subsubsection*{Ejercicio 13}
\begin{lstlisting}
-- Se define igual que los anteriores con la particularidad de que el mapper utiliza pattern matching.
-- El mapper se encargara de mappear un 1 en el pais de cada uno de los monumentos mientras que
-- ignorara las estructuras de tipo ciudad o calle. El reducer se encargara de combinar sumar
-- todas las apariciones devolviendo el valor deseado.
monumentosPorPais :: [(Structure, Dict String String)] -> [(String, Int)]
monumentosPorPais =  \lin -> mapReduce mapper reducer lin
	where mapper (Monument, dict) = [(dict!"country", 1)]
	      mapper (City, dict) = []
	      mapper (Street, dict) = []
	      mapper _ = error "fst(argumento) no es un Structure"
	      reducer = \(pais, l) -> [(pais, (length l))] 



-- ------------------------ Ejemplo de datos del ejercicio 13 ----------------------
data Structure = Street | City | Monument deriving Show

isMonument:: Structure -> Bool
isMonument Street = False
isMonument City = False
isMonument Monument = True

items :: [(Structure, Dict String String)]
items = [
    (Monument, [
      ("name","Obelisco"),
      ("latlong","-36.6033,-57.3817"),
      ("country", "Argentina")]),
    (Street, [
      ("name","Int. Guiraldes"),
      ("latlong","-34.5454,-58.4386"),
      ("country", "Argentina")]),
    (Monument, [
      ("name", "San Martin"),
      ("country", "Argentina"),
      ("latlong", "-34.6033,-58.3817")]),
    (City, [
      ("name", "Paris"),
      ("country", "Francia"),
      ("latlong", "-24.6033,-18.3817")]),
    (Monument, [
      ("name", "Bagdad Bridge"),
      ("country", "Irak"),
      ("new_field", "new"),
      ("latlong", "-11.6033,-12.3817")])
    ]


------------------------------------------------
------------------------------------------------
-- Funciones para el Test "lista de monumentos"
mapperListaMonus (Monument, dict) = [(1, dict!"name")]
mapperListaMonus (Street, _) = []
mapperListaMonus (City, _) = []
reducerListaMonus = \(k, l) -> l

mapperCuentaLetras = \s -> foldr (\e rec -> (e,[1]) : rec) [] s
reducerCuentaLetras = \(k, l) -> [(k, (length l))]

\end{lstlisting}
	\newpage

\end{document}

